%Author(s), Course variables
\newcommand{\titl}{Projektplan}
\newcommand{\courseno}{02121}
\newcommand{\course}{Indtroduktion til Softwareteknologi}
\newcommand{\lb}{\\}
%Basics
\documentclass[a4paper, danish]{article}
\usepackage[utf8]{inputenc}
\usepackage{babel}
\usepackage[moderate]{savetrees}
%Symbols and scientifics
\usepackage{amsmath, amsfonts, amssymb, bm}
\numberwithin{equation}{section}
\usepackage{physics}
\usepackage{mathtools}
\usepackage{siunitx}
\sisetup{
    per-mode = power ,
    round-mode = figures ,
    round-precision = 3 ,
    scientific-notation = false ,
    output-decimal-marker = {.} ,
    exponent-product = \times ,
    separate-uncertainty = true ,
    uncertainty-separator = \ ,
    range-phrase = - ,
    range-units =  single ,
    inter-unit-product = \ensuremath{{\cdot{}}} ,
    number-unit-product = \ ,
    multi-part-units = single ,
}

%Appendix, TOC and Bibliography
\usepackage{appendix}
\renewcommand\appendixtocname{Bilag}
\usepackage[nottoc]{tocbibind}
\setcounter{tocdepth}{2}
\usepackage{lastpage}

%Figures
\usepackage[svgnames]{xcolor} % Required to specify font color
\usepackage{float}
\usepackage{graphicx}
\usepackage{setspace}
\usepackage{subcaption}
\usepackage[format=plain,
    labelfont={bf,it,footnotesize},
    textfont={it,footnotesize}]{caption}
% \captionsetup[table]{name=Huskeord}
\captionsetup{font={stretch=0.9}}
\usepackage{wrapfig}
\usepackage[a4paper, centering, rmargin=2.5cm, tmargin=2.5cm, lmargin=2.5cm, bmargin=3.5cm]{geometry}
\usepackage{etoolbox}
\usepackage{verbatim}
\usepackage[space]{grffile}
\usepackage[final]{pdfpages}
\usepackage{array}
\usepackage{multirow}
\usepackage{dcolumn}
\usepackage{fontawesome}
\usepackage{tikz}
\usepackage{flowchart}
\usetikzlibrary{positioning, arrows}
\newcommand{\ttt}[1]{\texttt{#1}}
\newcommand{\F}{\mathtt{F}}
\newcommand{\T}{\mathtt{T}}
\usepackage{pgfgantt}
\usepackage{pdflscape}

\newcommand{\lorf}{\ensuremath{\lor\F}}
\newcommand{\lort}{\ensuremath{\lor\T}}
\newcommand{\landf}{\ensuremath{\land\F}}
\newcommand{\landt}{\ensuremath{\land\T}}
\newcommand{\tof}{\ensuremath{\to\!\F}}
\newcommand{\tot}{\ensuremath{\to\!\T}}
\newcommand{\lrf}{\ensuremath{\leftrightarrow\!\F}}
\newcommand{\lrt}{\ensuremath{\leftrightarrow\!\T}}
\newcommand{\negf}{\ensuremath{\neg\F}}
\newcommand{\negt}{\ensuremath{\neg\T}}
\newcommand{\allf}{\ensuremath{\forall\F}}
\newcommand{\allt}{\ensuremath{\forall\T}}
\newcommand{\exf}{\ensuremath{\exists\F}}
\newcommand{\ext}{\ensuremath{\exists\T}}

\newcommand{\first}[2]{\node (root) {\textcolor{red}{\scriptsize 1} \(#1\) : \texttt{#2}};}
\newcommand{\formula}[4]{\node (#1) [#2] {\textcolor{red}{\scriptsize #1} \(#3\) : \texttt{#4}};}
\newcommand{\branch}[2]{\path (#1) edge[-] (#2);}
\newcommand{\rbranch}[4]{\path (#3) edge[-] node [midway, right, blue] {\(#1\) på \(#2\)} (#4);}
\newcommand{\closed}[2]{\node (#1) [below = .1em of #2] {\(\times\)};}
\newcommand{\open}[2]{\node (#1) [below = .1em of #2] {\(\bigcirc\)};}
\newenvironment{tableau}{\begin{tikzpicture}[node distance = .5pt]}{\end{tikzpicture}}

%Header footer
\usepackage{fancyhdr}
\pagestyle{fancy}
\lhead{\titl \lb \courseno\ \course}
\chead{\includegraphics[width=.05\textwidth]{DTU}}
\rhead{Aslan Behbahani, Yahya Alwan \\ Abinav Aleti, Rasmus Wiuff}
\cfoot{Side \thepage\, af \pageref*{LastPage}}
\renewcommand{\headrulewidth}{0.4pt}
\renewcommand{\footrulewidth}{0.4pt}
\setlength{\headheight}{36.75034pt}

%Text tools
\usepackage{listings}
\usepackage{parcolumns}
\usepackage[super]{nth}
\usepackage[normalem]{ulem}
\usepackage{import}
\usepackage{url}
\usepackage{lipsum}
\usepackage{microtype}
\usepackage[pdfencoding=auto, psdextra]{hyperref}
\hypersetup{
    colorlinks   = true, %Colours links instead of ugly boxes
    urlcolor     = blue, %Colour for external hyperlinks
    linkcolor    = blue, %Colour of internal links
    citecolor   = red %Colour of citations
}
\usepackage[capitalise]{cleveref}
% \c        refname{table}{Huskeord}{Huskeord}
\usepackage{enumitem}
\newlist{arrowlist}{itemize}{1}
\setlist[arrowlist]{label={\(\rightarrow\)}}
\usepackage{booktabs}
\usepackage{todonotes}
\usepackage{silence}
\usepackage[square, longnamesfirst, numbers]{natbib}
\usepackage{empheq}
\usepackage{minted}
\setminted{fontsize=\small,
           linenos=true}
\usemintedstyle{tango}
\renewcommand{\listoflistingscaption}{Listings}
\newcommand{\im}[3]{\inputminted[linenos=true, python3=true, firstline=#2, lastline=#3]{python}{#1}}
\newcommand{\java}[3]{\inputminted[linenos=true, firstline=#2, lastline=#3]{java}{#1}}
\usepackage{tcolorbox}
\tcbuselibrary{skins,theorems}
\usepackage{pseudo}
\usepackage{tabularx}
\usepackage{tabto}
\TabPositions{1.5cm}
\pseudodefinestyle{fullwidth}{
    begin-tabular =
    \tabularx{\linewidth}[t]{@{}
        r
        >{\pseudosetup}
        X
        >{\leavevmode\small\color{blue}}
        p{0.5\linewidth}
        @{}},
    end-tabular =\endtabularx,
    setup-append = \RestorePseudoEq
}
\pseudoset{
    label=\small\arabic*,
    hd-space,
    ct-left = \hfill\texttt{*/},
    ct-right = \texttt{*/},
    ctfont = \texttt,
    fullwidth,
}
\newtcbtheorem[number within = section, crefname = {Algoritme}{algoritmer}]{algorithm}{Algoritme}{pseudo/tworuled, float}{alg}

%Definitions and new commands
\newcommand{\degr}{^{\circ}}
\newcommand{\me}{\mathrm{e}}
\newcommand*\mathinhead[2]{\texorpdfstring{\(\boldsymbol{#1}\)}{#2}}

%Title and sectioning
\def\Vhrulefill{\leavevmode\leaders\hrule height 0.7ex depth \dimexpr0.4pt-0.7ex\hfill\kern0pt}
\usepackage{titlesec}
\definecolor{DTUred}{cmyk}{0, .91, .72, .23}
\definecolor{FMNgrey}{cmyk}{.73,.43,.53,.38}
\newcolumntype{a}{>{\columncolor{gray}}c}
%Use letters insted of numbers in section numbering
% \renewcommand{\thesection}{\Alph{section}}
% \renewcommand{\thesubsection}{\Alph{subsection}}

\begin{document}

\titleformat{\section}[block]
{\normalfont\Large\scshape\filright\color{DTUred}}{\fbox{\thesection}}{1em}{}

\titleformat{\subsection}
{\titlerule
    \vspace{.8ex}%
    \normalfont\scshape\color{FMNgrey}}
{\thesubsection.}{.5em}{}

\title{\vspace{-4cm}\includegraphics[width=.10\textwidth]{DTU}\lb\vspace{.5em}\Huge\scshape\color{DTUred} \titl\lb\vspace{-4mm}\rule{4cm}{0.5mm}\lb\Large{\courseno \ \course}}
\author{af \textbf{s224819} \ Aslan Dalhoff Behbahani, \ \textbf{s224739} \ Yahya Alwan,\\\textbf{s224786} \ Abinav Reddy Aleti \ og \ \textbf{s163977} \ Rasmus Wiuff \\ \textbf{Gruppe 16}}
\date{4. januar 2023}
\maketitle

\pagenumbering{arabic}

\thispagestyle{empty}

\section{Problemanalyse}
Projektet deles i to problemstillinger: \emph{1}: Repræsentation af spillet i en GUI, og \emph{2}: Logik som følger reglerne i spillet. Vi benytter os af Model-View-Controller. Hvor Logikken varetages af Model og View.
\subsection{Model}
En klasse som indeholder et \emph{Board}-objekt med tilhørende logik for kontrol af spil-tilstande og lovlighed af brikkers placering/vending. Brættet er et heltals array med tilstande for tom, hvid og sort.
Til brættet findes metoder som kan placere, vende og kontrollere for lovlighed af placering, optælling af vundet terræn, etc.
\subsection{View}
View varetages af en klasse som konstruere GUI'en og opdateres af Controller klassen. Lytter efter events fra museklik og viser beskeder som guider spillerne.
\subsection{Controller}
Controller varetages af en klasse der holder styr på metodekald til brætspilklassen ud fra events og progression i spillet.
\subsection{Reversi regler}
Regler realiseres jf. \cref{tbl:logik}.
\begin{table}[H]
    \centering
    \caption{Reglerne for Reversi og deres implementering.}\label{tbl:logik}
    \begin{tabular}{ll}
        \toprule
        Problem             & Løsning                                                                  \\
        \midrule
        Startende spiller   & Controller tildeler en tilfældig farve til spiller 1 og 2.               \\
        Startkonfiguration  & Hvid starter med 2 brikker. Sort udfylder de resterende felter i midten. \\
        Placering af brik   & Metodekald til bræt-klassen afgør lovlighed.                             \\
                            & Besked til brugeren om evt. at prøve igen.                               \\
        Vending af brikker  & Metodekald til bræt-klassen afgør lovlighed.                             \\
                            & Besked til brugeren om evt. at prøve igen.                               \\
        Spillets afslutning & Hvis brættet er fyldt eller begge melder pas.                            \\
        Vinder              & Ved afslutning optælles vundet terræn.                                   \\
                            & Ved uafgjort genstartes spillet med ny startspiller.                     \\
        Genstart af spillet & Initialisere spillet med samme spillerfarver (ikke ligesom uafgjort).    \\
        \bottomrule
    \end{tabular}
\end{table}
\section{Tilvalg af avancerede tilføjelser}
\cref{tbl:avanceret} viser mulige tilføjelser til den avancerede version i prioriteret rækkefølge.
\begin{table}[H]
    \centering
    \caption{Mulige avancerede tilvalg}\label{tbl:avanceret}
    \begin{tabular}{llll}
        \toprule
        \emph{1:} Ikon                      & \emph{4:} Spillernavne           & \emph{7:} Visning af tid    & \emph{10:} Lyd              \\
        \emph{2:} Brikker vender automatisk & \emph{5:} Visning af mulige træk & \emph{8:} Speed-reversi     & \emph{11:} Highscore       \\
        \emph{3:} Fullscreen                & \emph{6:} Valg af farver         & \emph{9:} Visuelle effekter & \emph{12:} Gemme/hente spil \\
        \bottomrule
    \end{tabular}
\end{table}
\section{Ansvar, Kvalitetskontrol \& Tidsplan}
Vi laver løbende arbejdsfordelinger og laver daglig status. Efter \textbf{BasicReversi} og \textbf{AdvancedReversi} er der indlagt tid til uddybende kvalitetskontrol. Til udviklingen af \textbf{BasicReversi} vil de logiske komponenter udvikles sideløbende med GUI, indtil fokus bliver overført til GUI. Delopgaver under hele forløbet inkludere: test, Gradle behandling og rapportskrivning. Under implementering af \textbf{AdvancedReversi} kan forskellige branches af projektet fokusere på forskellige tilføjelser.
\subsection{Tidsplan}
\cref{fig:gantt} viser den indledningsvise plan.
\begin{figure}[H]
    \centering
    \caption{Gantt diagram over projektet}\label{fig:gantt}
    \begin{ganttchart}[
            expand chart = \textwidth,
            hgrid,
            vgrid,
            time slot format = isodate,
            % inline,
            % milestone inline label node/.append style={left=5mm},
            % milestone/.append style={xscale=.5},
        ]{2023-01-04}{2023-01-20}
        \gantttitlecalendar{month=shortname, day} \\
        \ganttgroup{BasicReversi}{2023-01-04}{2023-01-11} \\
        \ganttbar{Logisk model}{2023-01-04}{2023-01-06} \\
        \ganttbar{GUI implementering}{2023-01-04}{2023-01-10} \\
        \ganttbar{Test af BasicReversi}{2023-01-11}{2023-01-11} \\
        \ganttmilestone{BasicReversi færdig}{2023-01-11} \\
        \ganttgroup{AdvancedReversi}{2023-01-11}{2023-01-16} \\
        \ganttbar{Test af AdvancedReversi}{2023-01-17}{2023-01-18} \\
        \ganttmilestone{AdvancedReversi færdig}{2023-01-18} \\
        \ganttgroup{Rapportskrivning}{2023-01-04}{2023-01-20} \\
        \ganttbar{Rapportafslutning}{2023-01-19}{2023-01-20} \\
        \ganttmilestone{Aflevering}{2023-01-20}
    \end{ganttchart}
\end{figure}
\end{document}