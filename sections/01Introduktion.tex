% !TeX root = ..\G16.tex
\section{Introduktion \& Problemanalyse}
\subsection{Arbejdsfordeling}
Arbejdsfordelingen under projektet kan ses i \cref{sec:arbejde}.
\subsection{Opgaven}
I dette projekt har vi skulle skrive et program, der starter- og kører spillet "Reversi", et brætspil der handler om kontrollere et bræt med flest mulige brikker. Programmet skal være skrevet i JavaFX. Problemet består i at, via. diverse datastrukturer, at skulle beskrive et bræt, brikkerne såvel som deres egenskaber og opførsel. Der har skulle være en grænseflade, der giver udtryk for alt det visuelle, såvel som logikken, der binder de forskellige elementer sammen. 
Vi har som hjælp til at designe programmet benyttet os af model-view-control-strukturen. Vi har skrevet modellen, som er logikken bag alle reaktioner, programmet kan have til diverse handlinger, vi har skrevet view, som er det grafiske element, der bestemmer hvordan programmet skal se ud, former, farver mm. Til sidst har vi skrevet control, som har skulle specificere, hvilket handlinger, der skal reageres på. \newline
\newline
Et væsentligt problem, ved implementering af logik og det visuelle har været, at hvis har villet teste den ene, har det krævet den fuldstændige version af den anden. Hvis den del af gruppen, der har udviklet logikken, har villet teste om det virker, har de måtte udvikle en midlertidig løsning i form af en alternativ visualisering, da det grafiske stadig har været under udvikling.
\newline
\newline
Vores mål i dette projekt har været, at vi først og fremmest vil designe spillet i sin basisform, uden fejl og mangler. Spillet har skulle kunne køre, og været testet adskillige gange for robusthed og eventuelle små fejl. Dette har været vores minimumskrav, for at kunne kalde projektet "bestået".
Vi ville gerne skrive et program, som var let læseligt, overskueligt, og så simpelt som muligt i struktur, uden unødvendige besværligheder.
Derudover har vi villet gøre spillet så avanceret, som vi har kunnet. Dette betyder at vi har implementeret de valgfri tilføjelse, som vi idømte inde for vores faglige kapacitet. 
Et andet vigtigt mål har været at arbejde som en gruppe. Med det menes der at lære, hvorledes man arbejder sammen, deler arbejde op, når det kommer til et projekt af denne størrelse, og tage stilling til, hvordan at det kode, man er ved at skrive, kommer til at arbejde sammen med de andre gruppemedlemmers kode. 




\subsection{Specifikationer}
\subsection{Målsætninger}