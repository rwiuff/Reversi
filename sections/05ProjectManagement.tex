% !TeX root = ..\G16.tex
\section{Project Management}\label{sec:pm}
I følgende afsnit vil der blive diskuteret de valgte løsninger til at organisere projektet og løse opgaven.
\subsection{Tidsplan og tidsstyring}
Indledningsvist blev gruppen enige om en række principper og en tidsplan for at optimere udviklingsplanen. Principperne lyder:
\begin{enumerate}
    \item Nær-daglig status på underopgaver.
    \item Alle planer er tentative.
    \item Fælles udvikling af logik og algoritmer i vigtige metoder.
\end{enumerate}
Den oprindelige tidsplan kan ses i \cref{fig:gantt1}.
\begin{figure}[H]
    \centering
    \caption{Første tidsplan (fra Projektplanen)}\label{fig:gantt1}
    \begin{ganttchart}[
            % expand chart = \textwidth,
            hgrid,
            vgrid,
            time slot format = isodate,
            % inline,
            % milestone inline label node/.append style={left=5mm},
            % milestone/.append style={xscale=.5},
        ]{2023-01-04}{2023-01-20}
        \gantttitlecalendar{month=shortname, day} \\
        \ganttgroup{BasicReversi}{2023-01-04}{2023-01-11} \\
        \ganttbar{Logisk model}{2023-01-04}{2023-01-06} \\
        \ganttbar{GUI implementering}{2023-01-04}{2023-01-10} \\
        \ganttbar{Test af BasicReversi}{2023-01-11}{2023-01-11} \\
        \ganttmilestone{BasicReversi færdig}{2023-01-11} \\
        \ganttgroup{AdvancedReversi}{2023-01-11}{2023-01-16} \\
        \ganttbar{Test af AdvancedReversi}{2023-01-17}{2023-01-18} \\
        \ganttmilestone{AdvancedReversi færdig}{2023-01-18} \\
        \ganttgroup{Rapportskrivning}{2023-01-04}{2023-01-20} \\
        \ganttbar{Rapportafslutning}{2023-01-19}{2023-01-20} \\
        \ganttmilestone{Aflevering}{2023-01-20}
    \end{ganttchart}
\end{figure}
Under projektet overholdte vi ovennævnte principper, men tidsplanen ændredes markant. BasicReversi var færdig to dage over tid og rapportskrivningen blev lempeligt udskudt til den sidste uge.
\subsection{Arbejdsfordeling}
I forhold til arbejdsfordelingen er projektet opdelt i tre faser:
\begin{enumerate}
    \item Logik \& Brugerflade
    \item Controller klassen
    \item AdvancedReversi funktionaliteter
\end{enumerate}
I første fase arbejder gruppen i makkerpar. To personer nærstudere JavaFX og prøver at implementere den ønskede brugerflade og giver deres \emph{concerns} til det andet makkerpar. Det andet makkerpar arbejder på at implementere logikken bag brættet og brikkerne i en model klasse.\newline
I anden fase arbejder gruppen sammen på at løse forskellige problemstillinger. En kunne f.eks. arbejde på den overordnede struktur af metodekald i et spil Reversi, en anden arbejder på at oversætte museklik til koordinater, en tredje arbejder på reaktioner fra knapper, og en sidste på opdatering af det grafiske bræt ud fra den interne tilstand af bræt objektet. Til sidst en større sammenfletning som leder til en controller klasse, og efter test af spillet, et færdigt produkt.
I tredje fase bliver det oprindelige spil forgrenet via Git og hvert gruppemedlem implementerer et af gruppens tilføjelser af gangen. Når en implementation er færdig flettes denne med hovedgrenen. Her er der valgt en person som mestendels arbejder med at flette de forskellige grene sammen.
\subsection{Git, Github og Gradle}\label{sec:GGG}
Som projektstyringsværktøjer er der valgt følgende løsninger:
\begin{itemize}
    \item \faGit \ \href{https://git-scm.com/}{Git}: Til at kommunikere mellem lokale maskiner og \href{https://www.overleaf.com/}{Overleaf}, hvor rapporten kan skrives i fællesskab.
    \item \faGithub \ \href{https://github.com/}{Github}: Til at håndtere versionskontrol mellem medlemmer i gruppen.
    \item \href{https://gradle.org/}{Gradle}: Projektstyringsværktøj til at automatisere \emph{dependencies} (JavaFX), mappestrukturer og pakning af jar-filer.
\end{itemize}
\subsubsection{Git og Github}
Som nævnt i \cref{sec:introtilg} er projektet på Github \href{https://github.com/rwiuff/Reversi}{https://github.com/rwiuff/Reversi} \github{https://github.com/rwiuff/Reversi}. Dette \emph{repository} (repo) er udgangspunkt for versionskontrol i projektet. Desuden bruges Overleafs Git integration til at synkronisere projektet med lokale maskiner. Git som versionstyringsprogram har været essentielt under projektet. Der har været meget læring i at bruge \emph{branch} og \emph{merge} funktionerne til at kunne arbejde på forskelige aspekter af kode i de samme klasser på samme tid. Det har mestendels været uden de største problemer, men der er til stadighed meget mere at lære om samarbejde med VCS.
\subsubsection{Gradle}
Via Gradle kan mappestrukturen i \cref{fig:tree} genereres. Mappestrukturen indeholder desuden også bin og build mapper som bruges under kompilering og pakning af jar-arkiver, men disse er kun midlertidige instanser af det program man kører.
\begin{figure}[H]
    \caption{Mappestrukturen i JavaFX projekterne BasicReversi og AdvancedReversi}\label{fig:tree}
    \dirtree{%
        .1 BasicReversi / AdvancedReversi \ldots{} \begin{minipage}[t]{6cm} Rodmappen indeholder \emph{gradlewrapper} og \emph{settings.gradle} \end{minipage}.
        .2 app \ldots{} \begin{minipage}[t]{5cm} \emph{build.gradle} \end{minipage}.
        .3 src.
        .4 main.
        .5 java.
        .6 basicreversi / advancedreversi \ldots{} \begin{minipage}[t]{5cm} \mintinline{java}|.java| klasser \end{minipage}.
        .5 resources.
        .6 basicreversi / advancedreversi \ldots{} \begin{minipage}[t]{5cm} Billedfiler og \mintinline{java}|.fxml| filer \end{minipage}.
        .4 test.
        .5 java.
        .6 basicreversi / advancedreversi \ldots{} \begin{minipage}[t]{5cm} Testklasser \end{minipage}.
        .2 gradle.
        .3 wrapper \ldots{} \begin{minipage}[t]{5cm} Indeholder wrapper filer \end{minipage}.
        .2 releases \ldots{} \begin{minipage}[t]{10cm} Indeholde de færdige jar-arkiver \end{minipage}.
        }
\end{figure}

Gradle tilgodeser automatisk udførsel af skrevne tests, håndtering af kompilerede .class filer og pakning af jar arkiver. Desuden tilgodeser Gradle at dependencies er tilgængelige for miljøet der åbner projektet, uden at man skal importere JavaFX moduler manuelt. Er der en fejl i den version af JavaFX der bruges, kan man blot opdatere versionsnummeret i \emph{build.gradle} og herefter opdateres JavaFX.

