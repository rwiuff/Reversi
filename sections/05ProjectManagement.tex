% !TeX root = ..\G16.tex
\section{Project Management}\label{sec:pm}
I følgende afsnit vil der blive diskuteret de valgte løsninger til at organisere projektet og løse opgaven.
\subsection{Tidsplan og tidsstyring}
Indledningsvist blev gruppen enige om en række principper og en tidsplan for at optimere udviklingsplanen. Principperne lyder:
\begin{enumerate}
    \item Nær-daglig status på underopgaver.
    \item Overlap mellem test og implementation af funktioner.
    \item Alle planer er tentative.
    \item Fælles udvikling af logik og algoritmer i vigtige metoder.
\end{enumerate}
Den oprindelige tidsplan kan ses i \cref{fig:gantt1}.
\begin{figure}[H]
    \centering
    \caption{Første tidsplan (fra Projektplanen)}\label{fig:gantt1}
    \begin{ganttchart}[
            % expand chart = \textwidth,
            hgrid,
            vgrid,
            time slot format = isodate,
            % inline,
            % milestone inline label node/.append style={left=5mm},
            % milestone/.append style={xscale=.5},
        ]{2023-01-04}{2023-01-20}
        \gantttitlecalendar{month=shortname, day} \\
        \ganttgroup{BasicReversi}{2023-01-04}{2023-01-11} \\
        \ganttbar{Logisk model}{2023-01-04}{2023-01-06} \\
        \ganttbar{GUI implementering}{2023-01-04}{2023-01-10} \\
        \ganttbar{Test af BasicReversi}{2023-01-11}{2023-01-11} \\
        \ganttmilestone{BasicReversi færdig}{2023-01-11} \\
        \ganttgroup{AdvancedReversi}{2023-01-11}{2023-01-16} \\
        \ganttbar{Test af AdvancedReversi}{2023-01-17}{2023-01-18} \\
        \ganttmilestone{AdvancedReversi færdig}{2023-01-18} \\
        \ganttgroup{Rapportskrivning}{2023-01-04}{2023-01-20} \\
        \ganttbar{Rapportafslutning}{2023-01-19}{2023-01-20} \\
        \ganttmilestone{Aflevering}{2023-01-20}
    \end{ganttchart}
\end{figure}
Under projektet overholdte vi ovennævnte principper, men tidsplanen ændredes markant. BasicReversi var færdig to dage over tid og rapportskrivningen blev lempeligt udskudt til den sidste uge.
\subsection{Arbejdsfordeling}
I forhold til arbejdsfordelingen er projektet opdelt i tre faser:
\begin{enumerate}
    \item Logik \& Brugerflade
    \item Controller klassen
    \item AdvancedReversi funktionaliteter
\end{enumerate}
I første fase arbejder gruppen i makkerpar. To personer nærstudere JavaFX og prøver at implementere den ønskede brugerflade og giver deres \emph{concerns} til det andet makkerpar. Det andet makkerpar arbejder på at implementere logikken bag brættet og brikkerne i en model klasse.\newline
I anden fase arbejder gruppen sammen på at løse forskellige problemstillinger. En kunne f.eks. arbejde på den overordnede struktur af metodekald i et spil Reversi, en anden arbejder på at oversætte museklik til koordinater, en tredje arbejder på reaktioner fra knapper, og en sidste på opdatering af det grafiske bræt ud fra den interne tilstand af bræt objektet. Til sidst en større sammenfletning som leder til en controller klasse, og efter test af spillet, et færdigt produkt.
I tredje fase bliver det oprindelige spil forgrenet via Git og hvert gruppemedlem implementerer et af gruppens tilføjelser af gangen. Når en implementation er færdig flettes denne med hovedgrenen. Her er der valgt en person som mestendels arbejder med at flette de forskellige grene sammen.
\subsection{Git, Github og Gradle}\label{sec:GGG}
Som projektstyringsværktøjer er der valgt følgende løsninger:
\begin{itemize}
    \item \href{https://gradle.org/}{Gradle}: Projektstyringsværktøj til at automatisere \emph{dependencies} (JavaFX), mappestrukturer og pakning af jar-filer.
    \item \faGit \ \href{https://git-scm.com/}{Git}: Til at kommunikere mellem lokale maskiner og \href{https://www.overleaf.com/}{Overleaf}, hvor rapporten kan skrives i fællesskab.
    \item \faGithub \ \href{https://github.com/}{Github}: Til at håndtere versionskontrol mellem medlemmer i gruppen.
\end{itemize}