% !TeX root = ..\G16.tex
\section{Implementering}

\subsection{Board}

\subsubsection{Basic}

\subsubsection{Avanceret}

\subsection{Main}

\subsubsection{Basic}

\subsubsection{Avanceret}

\subsubsection{FXML}\label{BD}
Brættet er lavet af en 8x8 gridpane med en pane i hver celle. Hver pane indeholdt i gridpanes cellen er tildelt et unikt id. For eksempel er 05 et unikt id tildelt til en pane, hvor 0 repræsenterer rækkeindekset, og 5 repræsenterer kolonneindekset for panes placering i gridpane. Hver pane er tildelt et id for at gøre det nemmere at placere/tegne eller slette en brik fra brættet.


\subsection{Controller}

\subsubsection{Basic}



\subsubsection{Avanceret}
I den avancerede controller klasse er der rykket om på rækkefølgen, hvori metoderne kaldes.
\begin{table}[H]
\centering
\caption{}\label{tbl:}
\begin{tabular}{lllll}
\toprule
Metode & Kaldte metoder & Beskrivelse & Input & Output \\
\midrule
beginGame() & setName() & Sætter hele spillet i gang og loader spillebrættet & ActionEvent event & Ting 4 \\
setName() & setNameBtn() \\
& checkScore() \\
setNameBtn() & in() \\
in() & drawCircle() \\
firstFour() & update()\\
& checkscore() \\
onPaneClicked() & firstFour() \\
& hideLegalMoves() \\
& update()\\
& checkScore() \\
& showLegalMoves()\\
gameOver() & loadHighScore()\\
& checkHighScore() \\

\bottomrule
\end{tabular}
\end{table}