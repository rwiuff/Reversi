% !TeX root = ..\G16.tex
\section{Implementering}

\subsection{Board}\label{sec:boardImplement}
Modellen for brættet implementeres i \mintinline{java}|Board.java| klassen. \cref{tbl:boardfields} indeholder klassens felter.
\begin{table}[H]
    \centering
    \caption{Felter i \mintinline{java}|Board.java|}\label{tbl:boardfields}
    \begin{tabular}{ll}
        \toprule
        Felt                                                              & Formål                                           \\
        \midrule
        \mintinline{java}|int[][]board|                                   & Datastruktur der repræsenterer brættet.          \\
        \mintinline{java}|int turnCount|                                  & Turnummeret i spillet                            \\
        \mintinline{java}|int boardsize|                                  & Størrelsen \(n\) for et \(n\times n\) bræt.      \\
        \mintinline{java}|int forfeitCounter|                             & Antal gange en spillerne har meldt pas           \\
        \mintinline{java}|ArrayList<int[]> flipped|                       & Log over de sidste vendte brikker                \\
        \mintinline{java}|HashMap<Integer, String> players|               & Spillernes interne ID og navne                   \\
        \mintinline{java}|HashMap<String, HashMap<Integer,|               & Datastruktur for mulige træk for en given farve, \\
        \quad \mintinline{java}|HashMap<Integer, Integer[]>>> validMoves| & en given tur.                                    \\
        \bottomrule
    \end{tabular}
\end{table}
Brættet repræsenteres af en heltalsmatrix med værdierne \(0\) for tomt felt, \(1\) for hvid placering og \(2\) for sort placering. Ved at gemme turnumre og antal gange spillere har meldt pas kan board klassen afgøre hvis tur det er og om spillet er slut.
Loggen over vendte brikker endte med ikke at blive brugt, men formålet var at kontrollerklassen skulle vise hints om hvilke brikker der ville blive vendt ved et givent træk. Det blev stemt ned da det gør spillet for nemt.\newline
\cref{tbl:boardmethods} indeholder klassens metoder.
\begin{table}[H]
    \centering
    \caption{Metoder i \mintinline{java}|Board.java|}\label{tbl:boardmethods}
    \begin{tabular}{ll}
        \toprule
        Metode                                                        & Formål                                                                               \\
        \midrule
        \mintinline{java}|Board()|                                    & Konstruktør for brættet. Kalder \mintinline{java}|setPlayers()|.                     \\
        \mintinline{java}|getBoard()|                                 & Getter for board feltet.                                                             \\
        \mintinline{java}|setPiece(int row, int column, int colour)|  & Sætter manuelt brik, uanset lovlighed.                                               \\
                                                                      & Bruges til unit tests.                                                               \\
        \mintinline{java}|getTurn()|                                  & Getter for turnCount.                                                                \\
        \mintinline{java}|turnClock()|                                & Inrementere turnCount.                                                               \\
        \mintinline{java}|getPlayers()|                               & Getter for spiller hashmap.                                                          \\
        \mintinline{java}|getValidMoves()|                            & Getter for validMoves hashmap.                                                       \\
        \mintinline{java}|setPlayers()|                               & Tildeler tilfældigt spillere en farve.                                               \\
        \mintinline{java}|setPlayers(int id1, String player1,|        & Tildeler spillere specifikke farver.                                                 \\
        \quad \mintinline{java}|int id2, String player2)|             &                                                                                      \\
        \mintinline{java}|setPlayerName()|                            & Setter for specifikt spillernavn.                                                    \\
        \mintinline{java}|resetBoard()|                               & Nulstiller brættet, turnummer og pastæller.                                          \\
        \mintinline{java}|turnState(int colour)|                      & Kalder \mintinline{java}|moveAnalyser(colour)|,                                      \\
                                                                      & og udregner om der er ledige træk.                                                   \\
        \mintinline{java}|gameOver()|                                 & Afgør om spillet er slut hvis brættet er fyldt,                                      \\
                                                                      & eller der er meldt pas to gange i træk.                                              \\
        \mintinline{java}|initPlace(int row, int column, int colour)| & Indeholder logik for at udfylde startfelterne (midterste firkant).                   \\
        \mintinline{java}|place(int row, int column, int colour)|     & Placere en brik hvis placeringen er tom og indeholdt i validMoves.                   \\
                                                                      & Kalder \mintinline{java}|flip(move, colour)|.                                        \\
        \mintinline{java}|flip(String move, int colour)|              & Vender brikker ved et lovligt træk.                                                  \\
        \mintinline{java}|filled()|                                   & Afgører om brættet er fyldt.                                                         \\
        \mintinline{java}|checkWinner()|                              & Tæller brikker og afgører vinderen.                                                  \\
        \mintinline{java}|moveAnalyser(int colour)|                   & Gennemgår ledige felter og afgør og de er lovlige træk (se \cref{sec:moveAnalyser}). \\
        \mintinline{java}|findOwn(int[][] checkboard, int i,|         & Afgører om et træk resulterer i indkapsling af modstanderen.                         \\
        \quad \mintinline{java}|int j, int direction, int colour)|    &                                                                                      \\
        \mintinline{java}|saveFlips(int[] ownPiece, int i,|           & Gemmer indkapslede brikker.                                                          \\
        \quad \mintinline{java}|int j, int direction)|                &                                                                                      \\
        \bottomrule
    \end{tabular}
\end{table}

\subsection{Main}

\subsubsection{Basic}

\subsubsection{Avanceret}

\subsubsection{FXML}\label{BD}
Brættet er lavet af en 8x8 gridpane med en pane i hver celle. Hver pane indeholdt i gridpanes cellen er tildelt et unikt id. For eksempel er 05 et unikt id tildelt til en pane, hvor 0 repræsenterer rækkeindekset, og 5 repræsenterer kolonneindekset for panes placering i gridpane. Hver pane er tildelt et id for at gøre det nemmere at placere/tegne eller slette en brik fra brættet.


\subsection{Controller}

\subsubsection{Basic}



\subsubsection{Avanceret}
I den avancerede controller klasse er der rykket om på rækkefølgen, hvori metoderne kaldes. Nedenfor ses en tabel, der viser kronologien af de "overordnede" metoder", samt de metoder, de kalder.
\begin{table}[H]
\centering
\caption{}\label{tbl:}
\begin{tabular}{lll}
\toprule
Metode & Kalder metoderne: & evt. note \\
\midrule
beginGame() & setName() & Sætter hele spillet i gang\\
&& og loader spillebrættet \\
& \\

setName() & setNameBtn() & Beder spillerne om \\ 
& checkScore() & at indtaste deres navne \\
&& i tekstfelterne og \\
&& sætter deres score lig 0\\
& \\

setNameBtn() & in() & fastlåser spillernes navne \\
&& og sætter spillet i gang\\
& \\

in() & drawCircle() & tildeler spillerne deres \\
&& farver og viser de fire \\
&& første mulige træk \\
& \\

firstFour() & update() & Får spiller til at sætte \\
& checkscore() & de fire første brikker\\
& \\

onPaneClicked() & getRowIndex() & Styrer hvad der sker, når \\ 
& getColumnIndex() & der bliver klikket på et felt\\
&  firstFour() & Giver spillerne besked, om \\
& hideLegalMoves() &hvorvidt deres træk er lovlige\\
& update() & Melder pas, når der \\ 
& checkScore() & ikke er noget lovligt træk\\
& showLegalMoves() & Annoncerer spillets slutning,\\ 
&& når der ikke er \\
&& nogen lovlige træk tilbage\\
& \\

gameOver() & loadHighScore() & Erklærer vinderen\\
& checkHighScore() & og opdaterer high-score'n\\
&& (hvis den er blevet slået) \\


\bottomrule
\end{tabular}
\end{table}

\begin{table}[H]
\centering
\caption{Metoder i \mintinline{java}|Controller.java|}\label{tbl:2}

\begin{tabular}{lll}
\toprule
Metode & Kaldte metoder & Beskrivelse  \\
\midrule
restart(ActionEvent event) & reset() & Genstarter spillet og beder spillerene om   \\
& setName() & at indsatte deres navne  i tekstfelte igen.  \\
& & Spiller der lavet første træk i  \\
& & sidste runde, begynder anden \\
\\
reset() & & Iterer gennem brættet og rydder det \\
\\
update() & drawCircle() & Opdatere brættet og tegner circkler(brikker)  \\
& & på brættet ifølge Board objektet \\
\\
exitGame(ActionEvent event) & & lukker spille\\
\\
checkScore()& & Opdaterer scoren for begge spillere\\
\\

saveHighScore(int score,String name) & & Gemmer vinderens score og navnet adskilt af \\
& &  komma inde i "Highscore.txt" file, hvis vinderens  \\
& & score er højre end nuværende highscore\\
\\
loadHighScore() & & Læser highscore og spillersnavne fra \\
& & "Highscore.txt" filen og retunere de som arrayet.\\
\\
showHighScore(ActionEvent event) & loadHighScore() & Når man trykker på highscore-knappen, \\
& & vises en advarselsboks/-prompt, der viser highscore  \\
& & og spillerens navn, hvis highscore er indstillet. \\
\\

surrender(ActionEvent event) & & Annoncere overgivelsen af den spiller\\
& & der trykkede på knappen,\\
 & & og den anden spillers sejr\\
\\
getRowIndex(MouseEvent event)& & Returnerer rækkeindekset for pane, der er klikket på\\
\\
getColumnIndex(MouseEvent event)& & Returnerer søjleindekset for pane, der er klikket på\\
\\
showLegalMoves(int color) & drawCircle() & Tegner en cirkel hvor spiller kan mulig placer en brik\\
\\
hideLegalMoves() & & Rydder brættet for cirkler, hvilket indikerer mulige træk\\
\\
mainMenu(ActionEvent event) & & Returner tilbage til main-menu af spillet\\

\bottomrule
\end{tabular}
\end{table}
